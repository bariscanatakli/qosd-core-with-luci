% ====================================================================================
% Technical Whitepaper Template — qosd / OpenWrt / Telemetry / AI (Edge-Cloud)
% Author: Barış Can Ataklı (replace)
% License: CC-BY 4.0 (replace if needed)
% Compile: pdflatex whitepaper.tex (twice for ToC)
% ====================================================================================
\documentclass[11pt,a4paper]{article}

% ---- Encoding & Language (Turkish) ----
\usepackage[T1]{fontenc}
\usepackage[utf8]{inputenc}
\usepackage[turkish]{babel}

% ---- Layout ----
\usepackage[margin=2.5cm]{geometry}
\usepackage{setspace}
\setstretch{1.1}
\usepackage{titlesec}
\titleformat{\section}{\large\bfseries}{\thesection}{0.6em}{}
\titleformat{\subsection}{\normalsize\bfseries}{\thesubsection}{0.5em}{}
\usepackage{enumitem}
\setlist{nosep}

% ---- Graphics & Colors ----
\usepackage{graphicx}
\usepackage{float}
\usepackage{xcolor}
\definecolor{accent}{RGB}{2,108,223}
\definecolor{softgray}{gray}{0.2}
\usepackage{caption}
\captionsetup{labelfont=bf}

% ---- TikZ for architecture diagrams ----
\usepackage{tikz}
\usetikzlibrary{arrows.meta,positioning,shadows,fit}

% ---- Tables & Code ----
\usepackage{booktabs}
\usepackage{multirow}
\usepackage{listings}
\lstdefinestyle{code}{
  basicstyle=\ttfamily\small,
  frame=single,
  breaklines=true,
  tabsize=2,
  showstringspaces=false
}
\lstset{style=code}

% ---- References (simple) ----
% For quick use we embed a manual thebibliography. 
% For large projects, switch to biblatex/biber later.
\usepackage{hyperref}
\hypersetup{
  colorlinks=true,
  linkcolor=accent,
  urlcolor=accent,
  citecolor=accent
}

% ---- Title ----
\newcommand{\projecttitle}{Persona-Aware Dynamic QoS on OpenWrt: Edge–Cloud Telemetry and AI-Assisted Policy}
\newcommand{\shorttitle}{qosd Whitepaper}

\begin{document}

% ====================================================================================
% Title Page
% ====================================================================================
\begin{titlepage}
  \centering
  {\Large \textbf{\projecttitle}\par}
  \vspace{8pt}
  {\large Teknik Whitepaper (v0.1)\par}
  \vspace{18pt}
  \textbf{Yazar:} Barış Can Ataklı \hfill \textbf{Tarih:} \today

  \vfill
  \begin{minipage}{0.9\textwidth}
  \textbf{Özet} — Bu whitepaper, OpenWrt üzerinde geliştirilen \texttt{qosd} daemon'u ile
  persona-temelli dinamik QoS yönetimini; telemetri (Fluent Bit $\rightarrow$ OpenSearch) ve
  kenar (edge) üzerinde çalışan hafif yapay zekâ çıkarımı (inference) ile birleştiren uçtan uca
  mimariyi tanımlar. Sistem, CAKE tabanlı sıra yönetimi, çok katmanlı sınıflandırma, geçmiş veri
  analizleri ve politikaların gerçek zamanlı adaptasyonu ile düşük gecikme/kararlılık hedefler.
  \end{minipage}

  \vspace{12pt}
  \textbf{Anahtar Kelimeler:} OpenWrt, QoS, CAKE, Edge AI, Telemetry, OpenSearch, Fluent Bit, DSCP

  \vfill
  \textcolor{softgray}{\shorttitle} \\
  \textcolor{softgray}{Bu belge, proje tez/raporuna temel olacak teknik çerçeveyi sunar.}
\end{titlepage}

% ====================================================================================
\tableofcontents
\newpage

% ====================================================================================
\section{Giriş}
Tipik ev/ofis ağlarında, oyun/streaming/VoIP/bulk gibi farklı trafik sınıflarının aynı anda
adil ve gecikme-duyarlı yönetimi zordur. OpenWrt ekosistemi, CAKE/fq\_codel gibi AQM
yaklaşımları ile bufferbloat'ı azaltır; ancak \emph{statik} kurulumların günümüz trafiğine
uyumu sınırlıdır. Bu whitepaper, \texttt{qosd} ile \textbf{persona-aware}, \textbf{telemetry-driven} ve
\textbf{AI-assisted} bir QoS mimarisini ortaya koyar.

\paragraph{Hedefler:}
\begin{itemize}
  \item Gecikme/jitter düşüşü ve stabil DSCP temelli önceliklendirme,
  \item Gerçek-zaman \emph{ve} geçmiş verilerle politika adaptasyonu,
  \item Düşük kaynaklı router üzerinde çalışabilen \emph{quantized} AI çıkarımı,
  \item LuCI/standalone dashboard ile gözlemlenebilirlik ve kontrol.
\end{itemize}

% ====================================================================================
\section{Sistem Genel Bakış ve Gereksinimler}
\subsection{Bileşenler}
\begin{itemize}
  \item \textbf{qosd-core (router)}: Trafik sınıflandırma, DSCP işaretleme, CAKE/nftables uygulama.
  \item \textbf{Telemetry Agent (router)}: Fluent Bit ile JSON log $\rightarrow$ OpenSearch.
  \item \textbf{Collector/API (sunucu)}: REST endpoint, indeksleme, politika yayılımı.
  \item \textbf{OpenSearch/Kibana (sunucu)}: Kalıcı metrik depolama, analiz, dashboard.
  \item \textbf{AI Model Server (sunucu/edge)}: Eğitim sunucuda; çıkarım router’da (TFLite).
  \item \textbf{LuCI/Standalone UI}: Yerel yapılandırma, canlı grafikler, geçmiş sorguları.
\end{itemize}

\subsection{Fonksiyonel Gereksinimler (özet)}
\begin{enumerate}[label=FR\arabic*:]
  \item Persona atama (Gaming/Streaming/VoIP/Bulk/IoT).
  \item Çok-katmanlı sınıflandırma (port, akış istatistiği, DNS ipuçları).
  \item Adaptif politika: gecikme eşikleri, minimum garanti, boost/throttle.
  \item Gerçek-zaman ve tarihsel telemetri; dashboard görselleştirme.
\end{enumerate}

\subsection{Kısıtlar ve Hedef Donanım}
\begin{itemize}
  \item Hedef: MT7981B (AX3000T), 1.3\,GHz ARM64, 256\,MB RAM.
  \item QoS ek yük hedefi $<\!20\%$ CPU, $<\!64$\,MB RAM.
  \item Telemetry agent: 4–8\,MB RAM (Fluent Bit).
\end{itemize}

% ====================================================================================
\section{Mimari}
\subsection{Uçtan Uca Veri Akışı}
\begin{figure}[H]
  \centering
  \begin{tikzpicture}[
    node distance=1.5cm,
    box/.style={draw, rounded corners, align=center, fill=white, drop shadow, minimum width=3.6cm, minimum height=1.2cm},
    arrow/.style={-Latex, thick, color=accent}
  ]
    \node[box] (clients) {İstemciler\\(PC/TV/IoT)};
    \node[box, right=2.2cm of clients] (qosd) {\textbf{qosd-core}\\CAKE/DSCP};
    \node[box, right=2.2cm of qosd] (agent) {Fluent Bit\\(Tail JSON)};
    \node[box, right=2.2cm of agent] (collector) {Collector/API\\(REST Ingest)};
    \node[box, below=1.5cm of collector] (os) {OpenSearch\\+ Kibana};
    \node[box, above=1.5cm of collector] (ai) {AI Model Server\\(Train/Export)};
    \node[box, below=1.5cm of qosd] (luci) {LuCI/UI\\(Yerel Kontrol)};

    \draw[arrow] (clients) -- node[above]{Trafik} (qosd);
    \draw[arrow] (qosd) -- node[above]{JSON Metrikler} (agent);
    \draw[arrow] (agent) -- node[above]{HTTP POST} (collector);
    \draw[arrow] (collector) -- (os);
    \draw[arrow] (collector) -- (ai);
    \draw[arrow] (ai) |- node[pos=0.25, right]{Politika Önerileri} (qosd);
    \draw[arrow] (qosd) -- (luci);
    \draw[arrow] (os) -| node[pos=0.25, left]{Geçmiş Sorgu} (luci);
  \end{tikzpicture}
  \caption{Uçtan uca mimari ve veri akışı.}
\end{figure}

\subsection{QoS Kontrol Döngüsü}
\begin{enumerate}
  \item \textbf{Ölçüm}: RTT/jitter/throughput, persona ve DSCP istatistikleri toplanır.
  \item \textbf{Analiz}: Eşikler ve hedefler ile kıyaslanır (ör. gaming RTT $<\!35$\,ms).
  \item \textbf{Karar}: Min-garanti/boost/throttle ayarları belirlenir.
  \item \textbf{Uygulama}: CAKE tin/flow parametreleri ve nftables mark'ları güncellenir.
  \item \textbf{Doğrulama}: Etki metriklere yansır; döngü kararlılık için histerezis içerir.
\end{enumerate}

% ====================================================================================
\section{Uygulama Tasarımı}
\subsection{UCI Konfigürasyon (örnek)}
\begin{lstlisting}[language=bash,caption={/etc/config/qosd örnek şema}]
config persona 'gaming_pc'
  option mac 'AA:BB:CC:DD:EE:FF'
  option type 'gaming'
  option priority 'high'
  option bandwidth_share '30'

config policy 'default'
  option adaptive '1'
  option gaming_latency_target '35'
  option streaming_min_share '25'
\end{lstlisting}

\subsection{ubus API (öneri)}
\begin{lstlisting}[language=json,caption={ubus API yüzeyi (öneri)}]
{
  "qos.status": { "description": "Live QoS status/metrics" },
  "qos.personas": { "methods": ["list","add","update","delete"] },
  "qos.policies": { "methods": ["get_active","set_active","boost_device"] }
}
\end{lstlisting}

\subsection{Telemetry Log Formatı}
\begin{lstlisting}[language=json,caption={qosd JSON telemetri satırı (newline-delimited)}]
{"ts":"2025-10-14T19:10:00Z","device_mac":"AA:BB:..","persona":"gaming",
 "latency_ms":27.3,"jitter_ms":6.8,"bandwidth_kbps":8420,"dscp":"EF"}
\end{lstlisting}

% ====================================================================================
\section{Dağıtım ve Operasyon}
\subsection{Router (OpenWrt)}
\begin{itemize}
  \item \texttt{opkg install qosd luci-app-qosd fluent-bit}
  \item \texttt{/etc/init.d/qosd enable \&\& /etc/init.d/qosd start}
  \item \texttt{/etc/init.d/fluent-bit enable \&\& /etc/init.d/fluent-bit start}
\end{itemize}

\subsection{Sunucu (OpenSearch + Dashboards)}
\begin{itemize}
  \item Tek düğüm OpenSearch, 9200/tcp; Dashboards (Kibana eşleniği) 5601/tcp.
  \item Collector/API: FastAPI/Flask (opsiyonel), ya da Fluent Bit doğrudan OpenSearch HTTP output.
\end{itemize}

\subsection{Güvenlik}
\begin{itemize}
  \item Router $\rightarrow$ Sunucu trafiği: HTTPS + Token.
  \item Sunucuda RBAC/Index-level security (üretimde \emph{security plugin} açık).
\end{itemize}

% ====================================================================================
\section{Performans Bütçesi ve Hedefler}
\begin{table}[H]
  \centering
  \begin{tabular}{@{}lcc@{}}
    \toprule
    \textbf{Bileşen} & \textbf{CPU} & \textbf{RAM} \\
    \midrule
    qosd-core (idle) & $<\!5\%$ & $<\!10$\,MB \\
    qosd-core (peak) & $<\!20\%$ & $<\!32$\,MB \\
    Fluent Bit       & $<\!5\%$  & 4–8\,MB \\
    \bottomrule
  \end{tabular}
  \caption{Hedef kaynak kullanımı (AX3000T / MT7981B).}
\end{table}

% ====================================================================================
\section{Değerlendirme Planı (Özet)}
\subsection{Senaryolar}
\begin{enumerate}
  \item Oyun altında yük (eşzamanlı streaming + bulk).
  \item Video konferans önceliği (Zoom/Teams) + arka plan backup.
  \item Karışık hane kullanımı (gaming/streaming/web/IoT).
\end{enumerate}

\subsection{Metrikler}
RTT, jitter, paket kaybı, persona başına throughput, adaptasyon süresi, CAKE tin istatistikleri.

% ====================================================================================
\section{Yol Haritası}
\begin{enumerate}
  \item \textbf{MVP}: Persona $\rightarrow$ CAKE eşleme, temel ubus.
  \item \textbf{Beta}: Telemetry pipeline, dashboard grafikler.
  \item \textbf{RC}: Edge AI çıkarımı (TFLite), adaptif eşik öğrenimi.
  \item \textbf{Prod}: Çoklu router (controller), güvenlik sertleştirme.
\end{enumerate}

% ====================================================================================
\section{Sonuç}
Bu mimari, düşük maliyetli yönlendiricilerde dahi \emph{telemetry-aware} ve \emph{AI-assisted}
bir QoS sağlar. Kenarda (edge) çalışan çıkarım, bağlantı olmasa bile yerel kararı mümkün kılar;
sunucudaki geçmiş veriler ise politika optimizasyonunu sürekli iyileştirir.

% ====================================================================================
\appendix
\section*{Ekler}

\section{Fluent Bit Konfigürasyonu (OpenWrt)}
\begin{lstlisting}[language=ini,caption={/etc/fluent-bit.conf}]
[SERVICE]
  Flush        5
  Daemon       Off
  Log_Level    info

[INPUT]
  Name         tail
  Path         /var/log/qosd_metrics.log
  Tag          qosd
  Parser       json
  Refresh_Interval 2

[OUTPUT]
  Name         http
  Match        qosd
  Host         192.168.1.10      # OpenSearch host
  Port         9200
  URI          /qosd-metrics/_doc/
  Format       json_stream
  Retry_Limit  False
\end{lstlisting}

\section{OpenSearch Docker Compose (Sunucu)}
\begin{lstlisting}[language=yaml,caption={docker-compose.yml}]
version: '3.9'
services:
  opensearch:
    image: opensearchproject/opensearch:2.13.0
    container_name: opensearch
    environment:
      - discovery.type=single-node
      - plugins.security.disabled=true
    ports:
      - "9200:9200"
    volumes:
      - ./os-data:/usr/share/opensearch/data

  dashboards:
    image: opensearchproject/opensearch-dashboards:2.13.0
    container_name: dashboards
    environment:
      - OPENSEARCH_HOSTS=["http://opensearch:9200"]
    ports:
      - "5601:5601"
    depends_on:
      - opensearch
\end{lstlisting}

\section{UCI Örneği}
\begin{lstlisting}[language=bash]
config persona 'stream_tv'
  option mac '11:22:33:44:55:66'
  option type 'streaming'
  option priority 'medium'

config policy 'gaming_household'
  option adaptive '1'
  option gaming_latency_target '30'
  option bulk_max_share '10'
\end{lstlisting}

\section{Makefile Parçası (OpenWrt Paket)}
\begin{lstlisting}[language=make]
define Package/qosd
  SECTION:=net
  CATEGORY:=Network
  SUBMENU:=QoS
  TITLE:=Persona-aware QoS daemon
  DEPENDS:=+libubus +libuci +libjson-c
endef
\end{lstlisting}

% ====================================================================================
\begin{thebibliography}{9}
\bibitem{codel}
K. Nichols and V. Jacobson, ``Controlling Queue Delay,'' \emph{Communications of the ACM}, 55(7), 2012.

\bibitem{diffserv}
S. Blake et al., ``An Architecture for Differentiated Services,'' \emph{RFC 2475}, 1998.

\bibitem{ml-traffic}
T. T. T. Nguyen and G. Armitage, ``A Survey of Techniques for Internet Traffic Classification using ML,'' \emph{IEEE Comms Surveys}, 2008.

\bibitem{cake}
OpenWrt Project, ``CAKE \& SQM Scripts,'' Online Docs.

\bibitem{ebpf-edge}
M. Popescu et al., ``eBPF for Adaptive Network Telemetry on Edge Routers,'' \emph{IEEE Access}, 2024.

\bibitem{opensearch}
OpenSearch Project Docs, ``HTTP Bulk \& Index APIs,'' 2024.
\end{thebibliography}

\end{document}
